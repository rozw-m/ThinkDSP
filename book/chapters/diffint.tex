\chapter{Differentiation and Integration}
\label{diffint}

This chapter picks up where the previous chapter left off,
looking at the relationship between windows in the time domain
and filters in the frequency domain.

In particular, we'll look at the effect of a finite difference
window, which approximates differentiation, and the cumulative
sum operation, which approximates integration.

The code for this chapter is in {\tt chap09.ipynb}, which is in the
repository for this book (see Section~\ref{code}).
You can also view it at \url{http://tinyurl.com/thinkdsp09}.


\section{Finite differences}
\label{diffs}

In Section~\ref{smoothing}, we applied a smoothing window to
the stock price of Facebook and found that a smoothing
window in the time domain corresponds to a low-pass filter in
the frequency domain.

In this section, we'll look at daily price changes and
see that computing the difference between successive elements,
in the time domain, corresponds to a high-pass filter.

Here's the code to read the data, store it as a wave, and compute its
spectrum.

\begin{verbatim}
	import pandas as pd
	
	names = ['date', 'open', 'high', 'low', 'close', 'volume']
	df = pd.read_csv('fb_1.csv', header=0, names=names)
	ys = df.close.values[::-1]
	close = thinkdsp.Wave(ys, framerate=1)
	spectrum = close.make_spectrum()
\end{verbatim}

This example uses Pandas to read the CSV file; the
result is a DataFrame, {\tt df}, with columns for the opening
price, closing price, and high and low prices.  I select the closing
prices and save them in a Wave object.
The framerate is 1 sample per day.

Figure~\ref{fig.diff_int1} shows
this time series and its spectrum.
Visually, the time series resembles Brownian noise (see
Section~\ref{brownian}).
And the spectrum looks like a straight
line, albeit a noisy one.  The estimated slope is -1.9,
which is consistent with Brownian noise.

\begin{figure}
	% diff_int.py
	\centerline{\includegraphics[height=2.5in]{figs/diff_int1.pdf}}
	\caption{Daily closing price of Facebook and the spectrum of this time
		series.}
	\label{fig.diff_int1}
\end{figure}

Now let's compute the daily price change using {\tt np.diff}:

\begin{verbatim}
	diff = np.diff(ys)
	change = thinkdsp.Wave(diff, framerate=1)
	change_spectrum = change.make_spectrum()
\end{verbatim}

Figure~\ref{fig.diff_int2} shows the resulting wave and its spectrum.
The daily changes resemble white noise, and the estimated slope of the
spectrum, -0.06, is near zero, which is what we expect for white
noise.

\begin{figure}
	% diff_int2.py
	\centerline{\includegraphics[height=2.5in]{figs/diff_int2.pdf}}
	\caption{Daily price change of Facebook and the spectrum of this time series.}
	\label{fig.diff_int2}
\end{figure}


\section{The frequency domain}

Computing the difference
between successive elements is the same as convolution with
the window {\tt [1, -1]}.
If the order of those elements seems backward,
remember that convolution reverses the window before applying it
to the signal.

We can see the effect of this operation in the frequency domain
by computing the DFT of the window.

\begin{verbatim}
	diff_window = np.array([1.0, -1.0])
	padded = thinkdsp.zero_pad(diff_window, len(close))
	diff_wave = thinkdsp.Wave(padded, framerate=close.framerate)
	diff_filter = diff_wave.make_spectrum()
\end{verbatim}

\begin{figure}
	% diff_int.py
	\centerline{\includegraphics[height=2.5in]{figs/diff_int3.pdf}}
	\caption{Filters corresponding to the diff and differentiate operators (left) and integration operator (right, log-$y$ scale).}
	\label{fig.diff_int3}
\end{figure}

Figure~\ref{fig.diff_int3} shows the result.  The finite difference
window corresponds to a high pass filter: its amplitude increases with
frequency, linearly for low frequencies, and then sublinearly after
that.  In the next section, we'll see why.


\section{Differentiation}
\label{effdiff}

The window we used in the previous section is a
numerical approximation of the first derivative, so the filter
approximates the effect of differentiation.

Differentiation in the time domain corresponds to a simple filter
in the frequency domain; we can figure out what it is with a little
math.

Suppose we have a complex sinusoid with frequency $f$:
%
\[ E_f(t) = e^{2 \pi i f t} \]
%
The first derivative of $E_f$ is
%
\[ \frac{d}{dt} E_f(t) = 2 \pi i f e^{2 \pi i f t} \]
%
which we can rewrite as
%
\[ \frac{d}{dt} E_f(t) = 2 \pi i f E_f(t) \]
%
In other words, taking the derivative of $E_f$ is the same
as multiplying by $2 \pi i f$, which is a complex number
with magnitude $2 \pi f$ and angle $\pi/2$.

We can compute the filter that corresponds to differentiation,
like this:

\begin{verbatim}
	deriv_filter = close.make_spectrum()
	deriv_filter.hs = PI2 * 1j * deriv_filter.fs
\end{verbatim}

I started with the spectrum of {\tt close}, which has the right
size and framerate, then replaced the {\tt hs} with $2 \pi i f$.
Figure~\ref{fig.diff_int3} (left) shows this filter; it is a straight line.

As we saw in Section~\ref{synthmat}, multiplying a complex sinusoid
by a complex number has two effects: it multiplies
the amplitude, in this case by $2 \pi f$, and shifts the phase
offset, in this case by $\pi/2$.

If you are familiar with the language of operators and eigenfunctions,
each $E_f$ is an eigenfunction of the differentiation operator, with the
corresponding eigenvalue $2 \pi i f$.  See
\url{http://en.wikipedia.org/wiki/Eigenfunction}.

If you are not familiar with that language, here's what it
means:

\newcommand{\op}{\mathcal{A}}

\begin{itemize}
	
	\item An operator is a function that takes a function and returns
	another function.  For example, differentiation is an operator.
	
	\item A function, $g$, is an eigenfunction of an operator, $\op$, if
	applying $\op$ to $g$ has the effect of multiplying the function by
	a scalar.  That is, $\op g = \lambda g$.
	
	\item In that case, the scalar $\lambda$ is the eigenvalue that
	corresponds to the eigenfunction $g$.
	
	\item A given operator might have many eigenfunctions, each with
	a corresponding eigenvalue.
	
\end{itemize}

Because complex sinusoids are eigenfunctions of the differentiation
operator, they are easy to differentiate.  All we have to do is
multiply by a complex scalar.

For signals with more than one
component, the process is only slightly harder:

\begin{enumerate}
	
	\item Express the signal as the sum of complex sinusoids.
	
	\item Compute the derivative of each component by multiplication.
	
	\item Add up the differentiated components.
	
\end{enumerate}

If that process sounds familiar, that's because it is identical
to the algorithm for convolution in Section~\ref{effconv}: compute
the DFT, multiply by a filter, and compute the inverse DFT.

{\tt Spectrum} provides a method that applies the differentiation
filter:

\begin{verbatim}
	# class Spectrum:
	
	def differentiate(self):
	self.hs *= PI2 * 1j * self.fs
\end{verbatim}

We can use it to compute the derivative of the Facebook time series:

\begin{verbatim}
	deriv_spectrum = close.make_spectrum()
	deriv_spectrum.differentiate()
	deriv = deriv_spectrum.make_wave()
\end{verbatim}

\begin{figure}
	% diff_int.py
	\centerline{\includegraphics[height=2.5in]{figs/diff_int4.pdf}}
	\caption{Comparison of daily price changes computed by
		{\tt np.diff} and by applying the differentiation filter.}
	\label{fig.diff_int4}
\end{figure}

Figure~\ref{fig.diff_int4} compares the daily price changes computed by
{\tt np.diff} with the derivative we just computed.
I selected the first 50 values in the time series so we can see the
differences more clearly.

The derivative is noisier, because it amplifies the high frequency
components more, as shown in Figure~\ref{fig.diff_int3} (left).  Also, the
first few elements of the derivative are very noisy.  The problem
there is that the DFT-based derivative is based on the assumption that
the signal is periodic.  In effect, it connects the last element in
the time series back to the first element, which creates artifacts at
the boundaries.

To summarize, we have shown:

\begin{itemize}
	
	\item Computing the difference between successive values in a signal
	can be expressed as convolution with a simple window.  The result is
	an approximation of the first derivative.
	
	\item Differentiation in the time domain corresponds to a simple
	filter in the frequency domain.  For periodic signals, the result is
	the first derivative, exactly.  For some non-periodic signals, it
	can approximate the derivative.
	
\end{itemize}

Using the DFT to compute derivatives is the basis of {\bf spectral
	methods} for solving differential equations (see
\url{http://en.wikipedia.org/wiki/Spectral_method}).

In particular, it is useful for the analysis of linear, time-invariant
systems, which is coming up in Chapter~\ref{systems}.


\section{Integration}

\begin{figure}
	% diff_int.py
	\centerline{\includegraphics[height=2.5in]{figs/diff_int5.pdf}}
	\caption{Comparison of the original time series and the integrated
		derivative.}
	\label{fig.diff_int5}
\end{figure}

In the previous section, we showed that differentiation in the time
domain corresponds to a simple filter in the frequency domain: it
multiplies each component by $2 \pi i f$.  Since integration is
the inverse of differentiation, it also corresponds to a simple
filter: it divides each component by $2 \pi i f$.

We can compute this filter like this:

\begin{verbatim}
	integ_filter = close.make_spectrum()
	integ_filter.hs = 1 / (PI2 * 1j * integ_filter.fs)
\end{verbatim}

Figure~\ref{fig.diff_int3} (right) shows this filter on a log-$y$ scale,
which makes it easier to see.

{\tt Spectrum} provides a method that applies the integration filter:

\begin{verbatim}
	# class Spectrum:
	
	def integrate(self):
	self.hs /= PI2 * 1j * self.fs
\end{verbatim}

We can confirm that the integration filter is correct by applying it
to the spectrum of the derivative we just computed:

\begin{verbatim}
	integ_spectrum = deriv_spectrum.copy()
	integ_spectrum.integrate()
\end{verbatim}

But notice that at $f=0$, we are dividing by 0.  The result in
NumPy is NaN, which is a special floating-point value that
represents ``not a number''.  We can partially deal with this
problem by setting this value to 0 before converting the
spectrum back to a wave:

\begin{verbatim}
	integ_spectrum.hs[0] = 0
	integ_wave = integ_spectrum.make_wave()
\end{verbatim}

Figure~\ref{fig.diff_int5} shows this integrated derivative along with
the original time series.  They are almost identical, but the
integrated derivative has been shifted down.  The problem is that when
we clobbered the $f=0$ component, we set the bias of the signal to 0.
But that should not be surprising; in general, differentiation loses
information about the bias, and integration can't recover it.  In some
sense, the NaN at $f=0$ is telling us that this element is unknown.

\begin{figure}
	% diff_int.py
	\centerline{\includegraphics[height=2.5in]{figs/diff_int6.pdf}}
	\caption{A sawtooth wave and its spectrum.}
	\label{fig.diff_int6}
\end{figure}

If we provide this ``constant of integration'', the results are
identical, which confirms that this integration filter is the correct
inverse of the differentiation filter.

\section{Cumulative sum}
\label{cumsum}

In the same way that the diff operator approximates differentiation,
the cumulative sum approximates integration.
I'll demonstrate with a Sawtooth signal.

\begin{verbatim}
	signal = thinkdsp.SawtoothSignal(freq=50)
	in_wave = signal.make_wave(duration=0.1, framerate=44100)
\end{verbatim}

Figure~\ref{fig.diff_int6} shows this wave and its spectrum.

{\tt Wave} provides a method that computes the cumulative sum of
a wave array and returns a new Wave object:

\begin{verbatim}
	# class Wave:
	
	def cumsum(self):
	ys = np.cumsum(self.ys)
	ts = self.ts.copy()
	return Wave(ys, ts, self.framerate)
\end{verbatim}

We can use it to compute the cumulative sum of \verb"in_wave":

\begin{verbatim}
	out_wave = in_wave.cumsum()
	out_wave.unbias()
\end{verbatim}

\begin{figure}
	% diff_int.py
	\centerline{\includegraphics[height=2.5in]{figs/diff_int7.pdf}}
	\caption{A parabolic wave and its spectrum.}
	\label{fig.diff_int7}
\end{figure}

Figure~\ref{fig.diff_int7} shows the resulting wave and its spectrum.
If you did the exercises in Chapter~\ref{harmonics}, this waveform should
look familiar: it's a parabolic signal.

Comparing the spectrum of the parabolic signal to the spectrum of the
sawtooth, the amplitudes of the components drop off more quickly.  In
Chapter~\ref{harmonics}, we saw that the components of the sawtooth
drop off in proportion to $1/f$.  Since the cumulative sum
approximates integration, and integration filters components in
proportion to $1/f$, the components of the parabolic wave drop off in
proportion to $1/f^2$.

We can see that graphically by computing the filter that corresponds
to the cumulative sum:

\begin{verbatim}
	cumsum_filter = diff_filter.copy()
	cumsum_filter.hs = 1 / cumsum_filter.hs
\end{verbatim}

Because {\tt cumsum} is the inverse operation of {\tt diff}, we
start with a copy of \verb"diff_filter", which is the filter
that corresponds to the {\tt diff} operation, and then invert the
{\tt hs}.

\begin{figure}
	% diff_int.py
	\centerline{\includegraphics[height=2.5in]{figs/diff_int8.pdf}}
	\caption{Filters corresponding to cumulative sum and integration.}
	\label{fig.diff_int8}
\end{figure}

Figure~\ref{fig.diff_int8} shows the filters corresponding to
cumulative sum and integration.  The cumulative sum is a good
approximation of integration except at the highest frequencies,
where it drops off a little faster.

To confirm that this is the correct filter for the cumulative
sum, we can compare it to the ratio of the spectrum
\verb"out_wave" to the spectrum of \verb"in_wave":

\begin{verbatim}
	in_spectrum = in_wave.make_spectrum()
	out_spectrum = out_wave.make_spectrum()
	ratio_spectrum = out_spectrum.ratio(in_spectrum, thresh=1)
\end{verbatim}

\begin{figure}
	% diff_int.py
	\centerline{\includegraphics[height=2.5in]{figs/diff_int9.pdf}}
	\caption{Filter corresponding to cumulative sum and actual ratios of
		the before-and-after spectrums.}
	\label{fig.diff_int9}
\end{figure}

And here's the method that computes the ratios:

\begin{verbatim}
	def ratio(self, denom, thresh=1):
	ratio_spectrum = self.copy()
	ratio_spectrum.hs /= denom.hs
	ratio_spectrum.hs[denom.amps < thresh] = np.nan
	return ratio_spectrum
\end{verbatim}

When {\tt denom.amps} is small, the resulting ratio is noisy,
so I set those values to NaN.

Figure~\ref{fig.diff_int9} shows the ratios and the filter
corresponding to the cumulative sum.  They agree, which confirms that
inverting the filter for {\tt diff} yields the filter for {\tt
	cumsum}.

Finally, we can confirm that the Convolution Theorem applies by
applying the {\tt cumsum} filter in the frequency domain:

\begin{verbatim}
	out_wave2 = (in_spectrum * cumsum_filter).make_wave()
\end{verbatim}

Within the limits of floating-point error, \verb"out_wave2" is
identical to \verb"out_wave", which we computed using {\tt cumsum}, so
the Convolution Theorem works!  But note that this demonstration only
works with periodic signals.


\section{Integrating noise}

In Section~\ref{brownian}, we generated Brownian noise by computing the
cumulative sum of white noise.
Now that we understand the effect of {\tt cumsum} in the frequency
domain, we have some insight into the spectrum of Brownian noise.

White noise has equal power at all frequencies, on average.  When we
compute the cumulative sum, the amplitude of each component is divided
by $f$.  Since power is the square of magnitude, the power of each
component is divided by $f^2$.  So on average, the power at frequency
$f$ is proportional to $1 / f^2$:
%
\[ P_f = K / f^2 \]
%
where $K$ is a constant that's not important.
Taking the log of both sides yields:
%
\[ \log P_f = \log K - 2 \log f \]
%
And that's why, when we plot the spectrum of Brownian noise on a
log-log scale, we expect to see a straight line with slope -2, at
least approximately.

In Section~\ref{diffs} we plotted the spectrum of closing prices for
Facebook, and estimated that the slope is -1.9, which is consistent
with Brownian noise.  Many stock prices have similar spectrums.

When we use the {\tt diff} operator to compute daily changes, we
multiplied the {\em amplitude} of each component by a filter proportional to
$f$, which means we multiplied the {\em power} of each component by $f^2$.
On a log-log scale, this operation adds 2 to the slope of the
power spectrum, which is why the estimated slope of the result
is near 0.1 (but a little lower, because {\tt diff} only approximates
differentiation).



\section{Exercises}

The notebook for this chapter is {\tt chap09.ipynb}.
You might want to read through it and run the code.

Solutions to these exercises are in {\tt chap09soln.ipynb}.

\begin{exercise}
	The goal of this exercise is to explore the effect of {\tt diff} and
	{\tt differentiate} on a signal. Create a triangle wave and plot
	it. Apply {\tt diff} and plot the result. Compute the spectrum of the
	triangle wave, apply {\tt differentiate}, and plot the result. Convert
	the spectrum back to a wave and plot it. Are there differences between
	the effect of {\tt diff} and {\tt differentiate} for this wave?
\end{exercise}

\begin{exercise}
	The goal of this exercise is to explore the effect of {\tt cumsum} and
	{\tt integrate} on a signal. Create a square wave and plot it. Apply
	{\tt cumsum} and plot the result. Compute the spectrum of the square
	wave, apply {\tt integrate}, and plot the result. Convert the spectrum
	back to a wave and plot it. Are there differences between the effect
	of {\tt cumsum} and {\tt integrate} for this wave?
\end{exercise}

\begin{exercise}
	The goal of this exercise is the explore the effect of integrating
	twice. Create a sawtooth wave, compute its spectrum, then apply {\tt
		integrate} twice. Plot the resulting wave and its spectrum. What is
	the mathematical form of the wave? Why does it resemble a sinusoid?
\end{exercise}

\begin{exercise}
	The goal of this exercise is to explore the effect of the 2nd
	difference and 2nd derivative. Create a {\tt CubicSignal}, which is
	defined in {\tt thinkdsp}. Compute the second difference by applying
	{\tt diff} twice. What does the result look like?  Compute the second
	derivative by applying {\tt differentiate} to the spectrum twice.
	Does the result look the same?
	
	Plot the filters that corresponds to the 2nd difference and the 2nd
	derivative and compare them. Hint: In order to get the filters on the
	same scale, use a wave with framerate 1.
\end{exercise}
\chapter{Harmonics}
\label{harmonics}

In this chapter I present several new waveforms; we will look at
their spectrums to understand their {\bf harmonic structure}, which is
the set of sinusoids they are made up of.

I'll also introduce one of the most important phenomena in digital
signal processing: aliasing.  And I'll explain a little more about how
the Spectrum class works.

The code for this chapter is in {\tt chap02.ipynb}, which is in the
repository for this book (see Section~\ref{code}).
You can also view it at \url{http://tinyurl.com/thinkdsp02}.


\section{Triangle waves}
\label{triangle}

A sinusoid contains only one frequency component, so its spectrum
has only one peak.  More complicated waveforms, like the
violin recording, yield DFTs with many peaks.  In this section we
investigate the relationship between waveforms and their spectrums.

\begin{figure}
	% aliasing.py
	\centerline{\includegraphics[height=2.5in]{figs/triangle-200-1.pdf}}
	\caption{Segment of a triangle signal at 200 Hz.}
	\label{fig.triangle.200.1}
\end{figure}

I'll start with a triangle waveform, which is like a straight-line
version of a sinusoid.  Figure~\ref{fig.triangle.200.1} shows a
triangle waveform with frequency 200 Hz.

To generate a triangle wave, you can use {\tt thinkdsp.TriangleSignal}:

\begin{verbatim}
	class TriangleSignal(Sinusoid):
	
	def evaluate(self, ts):
	cycles = self.freq * ts + self.offset / PI2
	frac, _ = np.modf(cycles)
	ys = np.abs(frac - 0.5)
	ys = normalize(unbias(ys), self.amp)
	return ys
\end{verbatim}

{\tt TriangleSignal} inherits \verb"__init__" from {\tt Sinusoid},
so it takes the same arguments: {\tt freq}, {\tt amp}, and {\tt offset}.

The only difference is {\tt evaluate}.  As we saw before,
{\tt ts} is the sequence of sample times where we want to
evaluate the signal.

There are many ways to generate a triangle wave.  The details
are not important, but here's how {\tt evaluate} works:

\begin{enumerate}
	
	\item {\tt cycles} is the number of cycles since the start time.
	{\tt np.modf} splits the number of cycles into the fraction
	part, stored in {\tt frac}, and the integer part, which is ignored
	\footnote{Using an underscore as a variable name is a convention that
		means, ``I don't intend to use this value.''}.
	
	\item {\tt frac} is a sequence that ramps from 0 to 1 with the given
	frequency.  Subtracting 0.5 yields values between -0.5 and 0.5.
	Taking the absolute value yields a waveform that zig-zags between
	0.5 and 0.
	
	\item {\tt unbias} shifts the waveform down so it is centered at 0; then
	{\tt normalize} scales it to the given amplitude, {\tt amp}.
	
\end{enumerate}

Here's the code that generates Figure~\ref{fig.triangle.200.1}:

\begin{verbatim}
	signal = thinkdsp.TriangleSignal(200)
	signal.plot()
\end{verbatim}

\begin{figure}
	% aliasing.py
	\centerline{\includegraphics[height=2.5in]{figs/triangle-200-2.pdf}}
	\caption{Spectrum of a triangle signal at 200 Hz, shown on two
		vertical scales.  The version on the right cuts off the fundamental
		to show the harmonics more clearly.}
	\label{fig.triangle.200.2}
\end{figure}

Next we can use the Signal to make a Wave, and use the Wave to
make a Spectrum:

\begin{verbatim}
	wave = signal.make_wave(duration=0.5, framerate=10000)
	spectrum = wave.make_spectrum()
	spectrum.plot()
\end{verbatim}

Figure~\ref{fig.triangle.200.2} shows two views of the result; the
view on the right is scaled to show the harmonics more clearly.  As
expected, the highest peak is at the fundamental frequency, 200 Hz,
and there are additional peaks at harmonic frequencies, which are
integer multiples of 200.

But one surprise is that there are no peaks at the even multiples:
400, 800, etc.  The harmonics of a triangle wave are all
odd multiples of the fundamental frequency, in this example
600, 1000, 1400, etc.

Another feature of this spectrum is the relationship between the
amplitude and frequency of the harmonics.  Their amplitude drops off
in proportion to frequency squared.  For example the frequency ratio
of the first two harmonics (200 and 600 Hz) is 3, and the amplitude
ratio is approximately 9.  The frequency ratio of the next two
harmonics (600 and 1000 Hz) is 1.7, and the amplitude ratio is
approximately $1.7^2 = 2.9$.  This relationship is called the
{\bf harmonic structure}.


\section{Square waves}
\label{square}

\begin{figure}
	% aliasing.py
	\centerline{\includegraphics[height=2.5in]{figs/square-100-1.pdf}}
	\caption{Segment of a square signal at 100 Hz.}
	\label{fig.square.100.1}
\end{figure}

{\tt thinkdsp} also provides {\tt SquareSignal}, which represents
a square signal.  Here's the class definition:

\begin{verbatim}
	class SquareSignal(Sinusoid):
	
	def evaluate(self, ts):
	cycles = self.freq * ts + self.offset / PI2
	frac, _ = np.modf(cycles)
	ys = self.amp * np.sign(unbias(frac))
	return ys
\end{verbatim}

Like {\tt TriangleSignal}, {\tt SquareSignal} inherits
\verb"__init__" from {\tt Sinusoid}, so it takes the same
parameters.

And the {\tt evaluate} method is similar.  Again, {\tt cycles} is
the number of cycles since the start time, and {\tt frac} is the
fractional part, which ramps from 0 to 1 each period.

{\tt unbias} shifts {\tt frac} so it ramps from -0.5 to 0.5,
then {\tt np.sign} maps the negative values to -1 and the
positive values to 1.  Multiplying by {\tt amp} yields a square
wave that jumps between {\tt -amp} and {\tt amp}.

\begin{figure}
	% aliasing.py
	\centerline{\includegraphics[height=2.5in]{figs/square-100-2.pdf}}
	\caption{Spectrum of a square signal at 100 Hz.}
	\label{fig.square.100.2}
\end{figure}

Figure~\ref{fig.square.100.1} shows three periods of a square
wave with frequency 100 Hz,
and Figure~\ref{fig.square.100.2} shows its spectrum.

Like a triangle wave, the square wave contains only odd harmonics,
which is why there are peaks at 300, 500, and 700 Hz, etc.
But the amplitude of the harmonics drops off more slowly.
Specifically, amplitude drops in proportion to frequency (not frequency
squared).

The exercises at the end of this chapter give you a chance to
explore other waveforms and other harmonic structures.


\section{Aliasing}
\label{aliasing}

\begin{figure}
	% aliasing.py
	\centerline{\includegraphics[height=2.5in]{figs/triangle-1100-2.pdf}}
	\caption{Spectrum of a triangle signal at 1100 Hz sampled at
		10,000 frames per second.  The view on the right is scaled to
		show the harmonics.}
	\label{fig.triangle.1100.2}
\end{figure}

I have a confession.  I chose the examples in the previous section
carefully to avoid showing you something confusing.  But now it's
time to get confused.

Figure~\ref{fig.triangle.1100.2} shows the spectrum of a triangle wave
at 1100 Hz, sampled at 10,000 frames per second.  Again, the view on
the right is scaled to show the harmonics.

The harmonics
of this wave should be at 3300, 5500, 7700, and 9900 Hz.
In the figure, there are peaks at 1100 and 3300 Hz, as expected, but
the third peak is at 4500, not 5500 Hz.  The
fourth peak is at 2300, not 7700 Hz.  And if you look closely, the
peak that should be at 9900 is actually at 100 Hz.  What's going on?

The problem is that when you evaluate the signal at
discrete points in time, you lose information about what happened
between samples.  For low frequency components, that's not a
problem, because you have lots of samples per period.

But if you sample a signal at 5000 Hz with 10,000 frames per second,
you only have two samples per period.  That turns out to be enough,
just barely, but if the frequency is higher, it's not.

To see why, let's generate cosine signals at 4500 and 5500 Hz,
and sample them at 10,000 frames per second:

\begin{verbatim}
	framerate = 10000
	
	signal = thinkdsp.CosSignal(4500)
	duration = signal.period*5
	segment = signal.make_wave(duration, framerate=framerate)
	segment.plot()
	
	signal = thinkdsp.CosSignal(5500)
	segment = signal.make_wave(duration, framerate=framerate)
	segment.plot()
\end{verbatim}

\begin{figure}
	% aliasing.py
	\centerline{\includegraphics[height=3.5in]{figs/aliasing1.pdf}}
	\caption{Cosine signals at 4500 and 5500 Hz, sampled at 10,000 frames
		per second.  The signals are different, but the samples are identical.}
	\label{fig.aliasing1}
\end{figure}

Figure~\ref{fig.aliasing1} shows the result.  I plotted the Signals
with thin gray lines and the samples using vertical lines,
to make it easier to compare the
two Waves.  The problem
should be clear: even though the Signals are different, the
Waves are identical!

When we sample a 5500 Hz signal at 10,000 frames per second, the
result is indistinguishable from a 4500 Hz signal.
For the same reason, a 7700 Hz signal is indistinguishable
from 2300 Hz, and a 9900 Hz is indistinguishable from 100 Hz.

This effect is called {\bf aliasing} because when the high frequency
signal is sampled, it appears to be a low frequency signal.

In this example, the highest frequency we can measure is 5000 Hz,
which is half the sampling rate.  Frequencies above 5000 Hz are folded
back below 5000 Hz, which is why this threshold is sometimes called
the ``folding frequency''.  It is sometimes also called the {\bf
	Nyquist frequency}.  See
\url{http://en.wikipedia.org/wiki/Nyquist_frequency}.

The folding pattern continues if the aliased frequency goes below
zero.  For example, the 5th harmonic of the 1100 Hz triangle wave is
at 12,100 Hz.  Folded at 5000 Hz, it would appear at -2100 Hz, but it
gets folded again at 0 Hz, back to 2100 Hz.  In fact, you can see a
small peak at 2100 Hz in Figure~\ref{fig.square.100.2}, and the next
one at 4300 Hz.


\section{Computing the spectrum}

We have seen the Wave method \verb"make_spectrum" several times.
Here is the implementation (leaving out some details we'll get
to later):

\begin{verbatim}
	from np.fft import rfft, rfftfreq
	
	# class Wave:
	def make_spectrum(self):
	n = len(self.ys)
	d = 1 / self.framerate
	
	hs = rfft(self.ys)
	fs = rfftfreq(n, d)
	
	return Spectrum(hs, fs, self.framerate)
\end{verbatim}

The parameter {\tt self} is a Wave object.  {\tt n} is the number
of samples in the wave, and {\tt d} is the inverse of the
frame rate, which is the time between samples.

{\tt np.fft} is the NumPy module that provides functions related
to the {\bf Fast Fourier Transform} (FFT), which is an efficient
algorithm that computes the Discrete Fourier Transform (DFT).

%TODO: add a forward reference to full fft
\verb"make_spectrum" uses {\tt rfft}, which stands for ``real
FFT'', because the Wave contains real values, not complex.  Later
we'll see the full FFT, which can handle complex signals.  The result
of {\tt rfft}, which I call {\tt hs}, is a NumPy array of complex
numbers that represents the amplitude and phase offset of each
frequency component in the wave.

The result of {\tt rfftfreq}, which I call {\tt fs}, is an array that
contains frequencies corresponding to the {\tt hs}.

To understand the values in {\tt hs}, consider these two ways to think
about complex numbers:

\begin{itemize}
	
	\item A complex number is the sum of a real part and an imaginary
	part, often written $x + iy$, where $i$ is the imaginary unit,
	$\sqrt{-1}$.  You can think of $x$ and $y$ as Cartesian coordinates.
	
	\item A complex number is also the product of a magnitude and a
	complex exponential, $A e^{i \phi}$, where $A$ is the {\bf
		magnitude} and $\phi$ is the {\bf angle} in radians, also called
	the ``argument''.  You can think of $A$ and $\phi$ as polar
	coordinates.
	
\end{itemize}

Each value in {\tt hs} corresponds to a frequency component: its
magnitude is proportional to the amplitude of the corresponding
component; its angle is the phase offset.

The Spectrum class provides two read-only properties, {\tt amps}
and {\tt angles}, which return NumPy arrays representing the
magnitudes and angles of the {\tt hs}.  When we plot a Spectrum
object, we usually plot {\tt amps} versus {\tt fs}.  Sometimes
it is also useful to plot {\tt angles} versus {\tt fs}.

Although it might be tempting to look at the real and imaginary
parts of {\tt hs}, you will almost never need to.  I encourage
you to think of the DFT as a vector of amplitudes and phase offsets
that happen to be encoded in the form of complex numbers.

To modify a Spectrum, you can access the {\tt hs} directly.
For example:

\begin{verbatim}
	spectrum.hs *= 2
	spectrum.hs[spectrum.fs > cutoff] = 0
\end{verbatim}

The first line multiples the elements of {\tt hs} by 2, which
doubles the amplitudes of all components.  The second line
sets to 0 only the elements of {\tt hs} where the corresponding
frequency exceeds some cutoff frequency.

But Spectrum also provides methods to perform these operations:

\begin{verbatim}
	spectrum.scale(2)
	spectrum.low_pass(cutoff)
\end{verbatim}

You can read the documentation of these methods and others at
\url{http://greenteapress.com/thinkdsp.html}.

At this point you should have a better idea of how the Signal, Wave,
and Spectrum classes work, but I have not explained how the Fast
Fourier Transform works.  That will take a few more chapters.


\section{Exercises}

Solutions to these exercises are in {\tt chap02soln.ipynb}.

\begin{exercise}
	If you use Jupyter, load {\tt chap02.ipynb} and try out the examples.
	You can also view the notebook at \url{http://tinyurl.com/thinkdsp02}.
\end{exercise}

\begin{exercise}
	A sawtooth signal has a waveform that ramps up linearly from -1 to 1,
	then drops to -1 and repeats. See
	\url{http://en.wikipedia.org/wiki/Sawtooth_wave}
	
	Write a class called
	{\tt SawtoothSignal} that extends {\tt Signal} and provides
	{\tt evaluate} to evaluate a sawtooth signal.
	
	Compute the spectrum of a sawtooth wave.  How does the harmonic
	structure compare to triangle and square waves?
\end{exercise}

\begin{exercise}
	Make a square signal at 1100 Hz and make a wave that samples it
	at 10000 frames per second.  If you plot the spectrum, you can
	see that most of the harmonics are aliased.
	When you listen to the wave, can you hear the aliased harmonics?
\end{exercise}


\begin{exercise}
	If you have a spectrum object, {\tt spectrum}, and print the
	first few values of {\tt spectrum.fs}, you'll see that they
	start at zero.  So {\tt spectrum.hs[0]} is the magnitude
	of the component with frequency 0.  But what does that mean?
	
	Try this experiment:
	
	\begin{enumerate}
		
		\item Make a triangle signal with frequency 440 and make
		a Wave with duration 0.01 seconds.  Plot the waveform.
		
		\item Make a Spectrum object and print {\tt spectrum.hs[0]}.
		What is the amplitude and phase of this component?
		
		\item Set {\tt spectrum.hs[0] = 100}.  Make a Wave from the
		modified Spectrum and plot it.  What effect does this operation
		have on the waveform?
		
	\end{enumerate}
	
\end{exercise}


\begin{exercise}
	Write a function that takes a Spectrum as a parameter and modifies
	it by dividing each element of {\tt hs} by the corresponding frequency
	from {\tt fs}.  Hint: since division by zero is undefined, you
	might want to set {\tt spectrum.hs[0] = 0}.
	
	Test your function using a square, triangle, or sawtooth wave.
	
	\begin{enumerate}
		
		\item Compute the Spectrum and plot it.
		
		\item Modify the Spectrum using your function and plot it again.
		
		\item Make a Wave from the modified Spectrum and listen to it.  What
		effect does this operation have on the signal?
		
	\end{enumerate}
	
\end{exercise}

\begin{exercise}
	Triangle and square waves have odd harmonics only; the sawtooth
	wave has both even and odd harmonics.  The harmonics of the
	square and sawtooth waves drop off in proportion to $1/f$; the
	harmonics of the triangle wave drop off like $1/f^2$.  Can you
	find a waveform that has even and odd harmonics that drop off
	like $1/f^2$?
	
	Hint: There are two ways you could approach this: you could
	construct the signal you want by adding up sinusoids, or you
	could start with a signal that is similar to what you want and
	modify it.
\end{exercise}

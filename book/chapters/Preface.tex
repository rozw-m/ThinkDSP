\chapter{Preface}
\label{preface}

Signal processing is one of my favorite topics.  It is useful
in many areas of science and engineering, and if you understand
the fundamental ideas, it provides insight into many things
we see in the world, and especially the things we hear.

But unless you studied electrical or mechanical engineering, you
probably haven't had a chance to learn about signal processing.  The
problem is that most books (and the classes that use them) present the
material bottom-up, starting with mathematical abstractions like
phasors.  And they tend to be theoretical, with few applications and
little apparent relevance.

The premise of this book is that if you know how to program, you
can use that skill to learn other things, and have fun doing it.

With a programming-based approach, I can present the most important
ideas right away.  By the end of the first chapter, you can analyze
sound recordings and other signals, and generate new sounds.  Each
chapter introduces a new technique and an application you can
apply to real signals.  At each step you learn how to use a
technique first, and then how it works.

This approach is more practical and, I hope you'll agree, more fun.


\section{Who is this book for?}

The examples and supporting code for this book are in Python.  You
should know core Python and you should be
familiar with object-oriented features, at least using objects if not
defining your own.

If you are not already familiar with Python, you might want to start
with my other book, {\it Think Python}, which is an introduction to
Python for people who have never programmed, or Mark
Lutz's {\it Learning Python}, which might be better for people with
programming experience.

I use NumPy and SciPy extensively.  If you are familiar with them
already, that's great, but I will also explain the functions
and data structures I use.

I assume that the reader knows basic mathematics, including complex
numbers.  You don't need much calculus; if you understand the concepts
of integration and differentiation, that will do.
I use some linear algebra, but I will explain it as we
go along.


\section{Using the code}
\label{code}

The code and sound samples used in this book are available from
\url{https://github.com/AllenDowney/ThinkDSP}.  Git is a version
control system that allows you to keep track of the files that
make up a project.  A collection of files under Git's control is
called a ``repository''.  GitHub is a hosting service that provides
storage for Git repositories and a convenient web interface.
\index{repository}
\index{Git}
\index{GitHub}

The GitHub homepage for my repository provides several ways to
work with the code:

\begin{itemize}
	
	\item You can create a copy of my repository
	on GitHub by pressing the {\sf Fork} button.  If you don't already
	have a GitHub account, you'll need to create one.  After forking, you'll
	have your own repository on GitHub that you can use to keep track
	of code you write while working on this book.  Then you can
	clone the repo, which means that you copy the files
	to your computer.
	\index{fork}
	
	\item Or you could clone
	my repository.  You don't need a GitHub account to do this, but you
	won't be able to write your changes back to GitHub.
	\index{clone}
	
	\item If you don't want to use Git at all, you can download the files
	in a Zip file using the button in the lower-right corner of the
	GitHub page.
	
\end{itemize}

All of the code is written to work in both Python 2 and Python 3
with no translation.

I developed this book using Anaconda from
Continuum Analytics, which is a free Python distribution that includes
all the packages you'll need to run the code (and lots more).
I found Anaconda easy to install.  By default it does a user-level
installation, not system-level, so you don't need administrative
privileges.  And it supports both Python 2 and Python 3.  You can
download Anaconda from \url{https://www.anaconda.com/distribution/}.
\index{Anaconda}

If you don't want to use Anaconda, you will need the following
packages:

\begin{itemize}
	
	\item NumPy for basic numerical computation, \url{http://www.numpy.org/};
	\index{NumPy}
	
	\item SciPy for scientific computation,
	\url{http://www.scipy.org/};
	\index{SciPy}
	
	\item matplotlib for visualization, \url{http://matplotlib.org/}.
	\index{matplotlib}
	
\end{itemize}

Although these are commonly used packages, they are not included with
all Python installations, and they can be hard to install in some
environments.  If you have trouble installing them, I
recommend using Anaconda or one of the other Python distributions
that include these packages.
\index{installation}

Most exercises use Python scripts, but some also use Jupyter
notebooks.  If you have not used Jupyter before, you can read about
it at \url{http://jupyter.org}.
\index{Jupyter}

There are three ways you can work with the Jupyter notebooks:

\begin{description}
	
	\item[Run Jupyter on your computer]
	
	If you installed Anaconda, you
	probably got Jupyter by default.  To check, start the server from
	the command line, like this:
	
	\begin{verbatim}
		$ jupyter notebook
	\end{verbatim}
	
	If it's not installed, you can install it in Anaconda like this:
	
	\begin{verbatim}
		$ conda install jupyter
	\end{verbatim}
	
	When you start the server, it should launch your default web browser
	or create a new tab in an open browser window.
	
	\item[Run Jupyter on Binder]
	
	Binder is a service that runs Jupyter in a virtual machine.  If you
	follow this link, \url{http://mybinder.org/repo/AllenDowney/ThinkDSP},
	you should get a Jupyter home page with the notebooks for this book
	and the supporting data and scripts.
	
	You can run the scripts and modify them to run your own code, but the
	virtual machine you run in is temporary.  Any changes you make will
	disappear, along with the virtual machine, if you leave it idle for
	more than about an hour.
	
	\item[View notebooks on nbviewer]
	
	When we refer to notebooks later in the book, we will provide links to
	nbviewer, which provides a static view of the code and results.  You
	can use these links to read the notebooks and listen to the examples,
	but you won't be able to modify or run the code, or use the
	interactive widgets.
	
\end{description}

Good luck, and have fun!



\section*{Contributor List}

If you have a suggestion or correction, please send email to
{\tt downey@allendowney.com}.  If I make a change based on your
feedback, I will add you to the contributor list
(unless you ask to be omitted).
\index{contributors}

If you include at least part of the sentence the
error appears in, that makes it easy for me to search.  Page and
section numbers are fine, too, but not as easy to work with.
Thanks!

\small

\begin{itemize}
	
	\item Before I started writing, my thoughts about this book
	benefited from conversations with Boulos Harb at Google and
	Aurelio Ramos, formerly at Harmonix Music Systems.
	
	\item During the Fall 2013 semester, Nathan Lintz and Ian Daniher
	worked with me on an independent study project and helped me with
	the first draft of this book.
	
	\item On Reddit's DSP forum, the anonymous user RamjetSoundwave
	helped me fix a problem with my implementation of Brownian Noise.
	And andodli found a typo.
	
	\item In Spring 2015 I had the pleasure of teaching this material
	along with Prof. Oscar Mur-Miranda and Prof. Siddhartan Govindasamy.
	Both made many suggestions and corrections.
	
	\item Silas Gyger corrected an arithmetic error.
	
	\item Giuseppe Masetti sent a number of very helpful suggestions.
	
	\item Eric Peters sent many helpful suggestions.
	
	% ENDCONTRIB
	
\end{itemize}


Special thanks to Freesound, which is the source of many of the
sound samples I use in this book, and to the Freesound users who
uploaded those sounds.  I include some of their wave files in
the GitHub repository for this book, using the original file
names, so it should be easy to find their sources.

Unfortunately, most Freesound users don't make their real names
available, so I can only thank them using their user names.  Samples
used in this book were contributed by Freesound users: iluppai,
wcfl10, thirsk, docquesting, kleeb, landup, zippi1, themusicalnomad,
bcjordan, rockwehrmann, marcgascon7, jcveliz.  Thank you all!

%100475__iluppai__saxophone-weep.wav
%http://www.freesound.org/people/iluppai/sounds/100475/

%105977__wcfl10__favorite-station.wav
%http://www.freesound.org/people/wcfl10/sounds/105977/

%120994__thirsk__120-oboe.wav
%http://www.freesound.org/people/Thirsk/sounds/120994/

%132736__ciccarelli__ocean-waves.wav
%http://www.freesound.org/people/ciccarelli/sounds/132736/

%180960__kleeb__gunshot.wav
%http://www.freesound.org/people/Kleeb/sounds/180960/

%18871__zippi1__sound-bell-440hz.wav
%http://www.freesound.org/people/zippi1/sounds/18871/

%253887__themusicalnomad__positive-beeps.wav
%http://www.freesound.org/people/themusicalnomad/sounds/253887/

%28042__bcjordan__voicedownbew.wav
%http://www.freesound.org/people/bcjordan/sounds/28042/

%72475__rockwehrmann__glissup02.wav
%http://www.freesound.org/people/rockwehrmann/sounds/72475/

%87778__marcgascon7__vocals.wav
%http://www.freesound.org/people/marcgascon7/sounds/87778/

%92002__jcveliz__violin-origional.wav
%http://www.freesound.org/people/jcveliz/sounds/92002/